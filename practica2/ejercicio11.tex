/*En un vacunatorio hay
un empleado de salud para vacunar a
50 personas. El empleado
de salud atiende a las personas de acuerdo con el orden de llegada y de a 5 personas a la
vez. Es decir, que cuando está libre debe esperar a que haya al menos 5 personas
esperando, luego vacuna a las 5 primeras personas, y al terminar las deja ir para esperar
por otras 5. Cuando ha atendido a las 50 personas el empleado de salud se retira.
Nota:
todos los procesos deben terminar su ejecución; asegurarse de no realizar Busy Waiting;
suponga que el empleado tienen una función VacunarPersona() que simula que el empleado
está vacunando a UNA persona.*/

cola esperaEnfermero;
sem mutexC = 1;
sem esperaLlamado[50] = ([50] 0)
sem esperaE = 0;

process Persona [id: 0..49]
{
    P(mutexC)
    Push(esperaEnfermero, id)
    V(mutexC)
    V(esperaE) //Aviso que llegué
    P(esperaLlamado[id]) //Espero a que seamos cinco
    SiendoVacunado();
}

process Empleado
{
    cant=0;
    while(cant<50)
    {
        for(i=0; i<5; i++) P(esperaE) //espero a los cinco
        for(i=0; i<5; i++)
        {
            P(mutexC)
            Pop(esperaEnfermero, idPersona)
            V(mutexC)
            V(esperaLlamado[idPersona])
            VacunarPersona(idPersona)
            cant++;
        }
    }
}


//podría hacerlo así para no despertarlos a todos en "distintos momentos"
process Empleado
{
    cant=0;
    while(cant<50)
    {
        idPersona[5];
        for(i=0; i<5; i++) P(esperaE) //espero a los cinco
        for(i=0; i<5; i++)
        {
            P(mutexC)
            Pop(esperaEnfermero, idPersona[i])
            V(mutexC)
        }
        for(i=0; i<5; i++){
            V(esperaLlamado[idPersona[i]])
            VacunarPersona(idPersona[i])
            cant++;
        }
    }
}